\documentclass[12pt]{article}

\usepackage{acronym,amsmath,circuitikz,tikz}
\usepackage{karnaugh-map}
\usetikzlibrary{shapes.geometric, arrows}


\title{Sistema Automação Veicular}
\date{}
\author{
  Thiago da Conceição\\
  \text{(Matricula - 1-2023126437)}
  \and
  Adelino Hoppe\\
  \text{(Matricula - 1-2023126141)}
  \and
  Emanuel Isaac Moura Martins\\
  \text{(Matricula - 1-2023126494)}
  \and
  Vinícius\\
  \text{Matricula -}
  \and
  João(Tarzan Asmatico)\\
  \text{Matricula -}
  \and
  Samuel Costa Mota\\
  \text{Matricula -}
} 

\begin{document}
\maketitle

\section*{Lista de Abreviações}
\begin{acronym}
\acro{SE}{- Sensor Esquerdo do Retrovisor}
\acro{SD}{- Sensor Direito do Retrovisor}
\acro{CV}{- Receptor do Comado de Voz}
\acro{S}{- Saida}
\end{acronym}

1) Tabela-Verdade\\

\begin{displaymath}
\begin{array}{|c c c| c}

SE & SD & CV & S\\

0 & 0 & 0 & 0\\
0 & 0 & 1 & 1\\
0 & 1 & 0 & 1\\
0 & 1 & 1 & 0\\
1 & 0 & 0 & 1\\
1 & 0 & 1 & 0\\
1 & 1 & 0 & 1\\
1 & 1 & 1 & 0\\
\end{array}
\end{displaymath}

2) Expressão Booleana\\
\begin{equation*}
\text{$\overline{SE} \cdot \overline{SD} \cdot CV+SD \cdot \overline{CV}+SE \cdot \overline{CV}$}
\end{equation*}

3) Mapa de Karnaugh\\

\begin{minipage}{0.4\textwidth}
\centering
\begin{karnaugh-map}[4][2][1][$SD\,CV$][$SE$]
\minterms{1,2,4,6}
\maxterms{0,3,5,7}
\end{karnaugh-map}
\end{minipage}

4) porta lógica\\
\newline

\begin{circuitikz} 
\draw (0,2) node[and port, number inputs=3](and1){};
\draw (and1.in 1) node[anchor=east] {$SE$};
\draw (and1.in 1) -| ++(0,0) node[left]{};
\node at (and1.bin 1) [ocirc, left]{};

\draw (and1.in 2) node[anchor=east] {$SD$};
\draw (and1.in 2) -| ++(0,0) node[left]{};
\node at (and1.bin 2) [ocirc, left]{};
\draw (and1.in 3) node[anchor=east] {$CV$};

\draw (0,0) node[and port, number inputs=2](and2){};
\draw (and2.in 1) node[anchor=east] {$SD$};
\draw (and2.in 2) node[anchor=east] {$CV$};

\draw (0,-2) node[and port, number inputs=2](and3){};
\draw (and3.in 1) node[anchor=east] {$SE$};
\draw (and3.in 2) node[anchor=east] {$CV$};
\node at (and3.bin 2) [ocirc, left]{};

\draw (5,0) node[or port, number inputs=3](or1){};
\draw
(and1.out) -- (or1.in 1)
(and2.out) -- (or1.in 2)
(and3.out) -- (or1.in 3)
(or1.out) node[anchor=west] {$S$};

\end{circuitikz}
\newline

5) fluxograma lógico\\
\newline


\end{document}